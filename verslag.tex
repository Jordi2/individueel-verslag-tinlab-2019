\documentclass{article}
\usepackage{graphicx} 
\usepackage[dutch]{babel}
\usepackage{graphicx}
\graphicspath{ {./images/} }
\begin{document}
\sffamily



\begin{titlepage}
  \centering
    \vfill
    {\bfseries\Huge
      Verslag Tinlab Advanced Algorithms \\
        \vskip2cm
      }
      {\bfseries\Large
        S. T. Udent\\
      }
      {
        \bfseries\normalsize
        176-671\\
        \vskip1cm
        \today\\
    }    
    \vfill
    \includegraphics[width=4cm]{logohr.png} % also works with logo.pdf
    \vfill
    \vfill
\end{titlepage}
\newpage
\tableofcontents

\newpage
\section{Inleiding}
In dit document wordt samengevat uitleg, voorbeelden en uitwerkingen van oefeningen beschreven. Deze samenvatting dient als leerboekje waar naar gerefereerd kan worden. 


royce waterfall method



Onderwerp(en): 
Nucleaire testen op het eiland Bikini Atoll.
vlucht 1951 mode confusion turkish airlines
kannon met dubbele loop, John Gilleland
Tsjernobly 1986
ariane 5 64 bit getal wrong format

Context:
bikini eilanden, bom die 5 megaton hogere yield had dan verwacht. vakkennis was er nog niet wat betreft physika om dit te kunnen voorspellen.


World vs. machine:



4 variablen methode:


Mode confusion 
%\cite{bredereke2002rigorous}, //geeft definitie van mode confusion.




\section{test}
test
\subsection{testing}
testing
\section{Requirements}

In dit hoofdstuk wordt er over requirements en specificaties aangeduid. De definities voor deze termen zijn die in dit document zullen worden aangehouden zijn die van \cite{thompson2000requirements}. \newline

\subsection{Requirements}

Requirements beschrijven fenomenen uit de buitenwereld die een probleem beschrijven. \newline \newline
Licht, tempratuur, mensen en andere al bestaande systemen zijn allemaal voorbeelden van fenomenen die door requirements beschreven kunnen worden. \newline

Mogelijke methode voor het vaststellen van requirments zijn: \newline
- vragen stellen / survey \newline
- observeren \newline
- brainstormen \newline
- ervaring / literatuurstudie \newline
- specialist raadplegen \newline


\subsection{specificaties}

Specificaties zijn de eisen waaraan de oplossing of product aan moet voldoen en mee wordt geëvalueerd. Software en hardware in voorbeelden van specificaties.

\section{Modellen}

\subsection{De Kripke structuur}

Kripke structuren zijn een manier om het gedrag van een systeem te modelleren. Belangrijke termen voor het bespreken van Kripke structuren zijn states en transities. \newline

Een state beschrijft de toestand waarin een systeem zich bevindt. In het geval dat de state niet duidelijk is voor de gebruiker dan wordt dat mode confusion genoemd. Transities zijn de overgangen van één state naar de anderen. 

\begin{figure}[!h]
	\centering
	\includegraphics[width=\textwidth]{kripke_structure_example1}
    \caption{Voorbeeld Kripke structuur}
	%\label{fig:figure2}
\end{figure}

\subsection{Soorten modellen}

\textbf{World vs. machine:}
requirements en vereiste. \newline \newline
\textbf{4 variabelen methode:}

Modellen voor computer systemen gebruiken vaak antropomorfisme analogie en intuïtieve conclusies wat tot lager precisie kan leiden. Het vier variabelen model helpt om software vereisten met groter precisie vast te stellen, door deze te modelleren op een manier zoals vaak door ingenieurs wordt toegepast. \cite{parnas1995functional} 

Zoals de naam suggereert bestaat het vier variabelen model uit vier onderdelen. Input, output, gecontroleerde variabelen en gemonitorde variabelen. Als eerste de gemonteerde variabelen die buiten het systeem bestaan en worden opgenomen door sensoren, waarna deze input is voor de software. Deze input draagt bij aan de output of acties die het systeem in de buitenwereld creëert, de resultaten hiervan heten de gecontroleerde variabelen. Zie ook Figuur ~\ref{fig:four_Variables} hieronder.


\begin{figure}[!h]
	\centering
	\includegraphics[width=\textwidth]{four_Variables}
    \caption{Vier variabelen model \cite{thompson2000requirements}}
	\label{fig:four_Variables}
\end{figure}

\subsection{Tijd}
//Behandeld in week 2 \newline
In Uppaal wordt tijd globaal op hetzelfde tempo bijgehouden. Tijd wordt in klokken bijgehouden en kan op elk moment worden uitgelezen of gereset. \cite{uppaalsmalltutorial}
\subsection{Guards en invarianten}
//Behandeld in week 2 \newline 
Guards zijn condities die op transities worden geplaatst, aan deze condities moet worden voldaan voordat deze transitie mogelijk is.\cite{uppaalsmalltutorial}


guards condities op transities, 

\subsection{Deadlock}

\subsection{Zeno gedrag}
//Behandeld in week 2
a\cite{leine2011zeno}
\section{Logica}

\subsection{Propositielogica}

\subsection{Predicatenlogica}

\subsection{Kwantoren}

\subsection{Dualiteiten}

\section{Computation tree logic}
//week 3
A Always E Exists F Eventually G Globally X Next

\subsection{De computation tree}

\subsection{Operator: AG}

\subsection{Operator: EG}

\subsection{Operator: AF}

\subsection{Operator: EF}

\subsection{Operator: AX}

\subsection{Operator: EX}

\subsection{Operator: p U q}

\subsection{Operator: p R q}

\subsection{Fairness}

\subsection{Liveness}

\newpage
\subsection{Temp log:}
Sheets:

inleiding w1
state transition diagrams w2
modellen w2
computation tree logic w3


week 1: 
world vs. machine
requirements en specifications |
systems engineering vs. software engineering
4 variablen |
week2:
tijd
guards en invarianten
deadlocks
week3:
logica
temporele logica (ctl)
(Computation tree logic
Linear tree logic
CTL*: combinatie van bovenstaande twee)
//zie dia voor example images van ctl logica
week4:




Wat is een goed model? cite the links from powerpoint, first links describes what the dia explains, deze link wordt ook gebruikt om de eindopdracht te beoordelen, eindverslag moet afweging laten zien wat betreft je keuzes, wat zijn je argumenten?

google scholar "to predict and serve?"

een goed model moet het verschil tussen ruis en echte informatie kunnen aantonen.

uppaal for sharing int below 5 don't use global variables

ebsilon/absilon-delta stelling wiskunde limiet begrip
\newpage
\bibliography{references}
\bibliographystyle{plain}
\end{document}


